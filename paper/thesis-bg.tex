\documentclass[12pt, oneside]{article} 

\usepackage{graphicx}
\usepackage{amssymb}
\usepackage[utf8]{inputenc}
\usepackage[bulgarian]{babel}

\usepackage{amsthm}
\theoremstyle{definition}
\newtheorem{definition}{Дефиниция}[section]
\newtheorem{theorem}{Теорема}[section]
\newtheorem{construction}{Конструкция}[section]
\newtheorem{example}{Пример}[section]
\newtheorem{proposition}{Твърдение}[section]
\newtheorem{corollary}{Следствие}[section]

\usepackage{geometry}
\geometry{letterpaper}

% \usepackage[pdf]{graphicx}
\usepackage{dot2texi}

\usepackage{tikz}
\usetikzlibrary{automata,positioning,arrows,decorations.pathreplacing,patterns}

% \setlength{\parskip}{1em}

\title{Лексически Анализ чрез Бимашини}

\begin{document}

\tableofcontents
\pagebreak

\section{Увод}

Основна задача на лексическият анализатор е да чете символите на входния текст, да ги групира под формата на лексеми (тоукъни) и извежда като изход редица от тези лексеми. Лексемата е структура от данни, която съдържа нейния тип, текст и позиция във входния текст. Тази информация в последствие се подава на парсер, който от своя страна извършва синтактичният анализ на текста.

В тази работа представяме разработка на алгоритъм, който по зададено множество от спецификации на лексеми, създава лексически анализатор. Анализаторът извлича лексимите за линейно време спрямо дължината на входния текст, използвайки бимашина.

Лексическите анализатори могат да се използват и за други цели освен идентификация на лексемите. Други приложения са премахване на сегменти от текст (като нови редове, интервали и пр.), броене на символи, или думи в текст, както и съобщаване на грешки.

\subsection{Мотиовация}

\subsection{Обобщение}

\pagebreak

\section{Основни дефиниции}

\subsection{Крайни автомати}

\begin{definition}
	\emph{Краен автомат} дефинираме като петорка \( \mathcal{A} = \langle \Sigma, Q, I, F, \Delta \rangle \), където

	\begin{itemize}
		\item \( \Sigma \) е \emph{крайна азбука от символи}
		\item \( Q \) е \emph{крайно множество от състояния}
		\item \( I \subseteq Q \) е \emph{множество от начални състояния}
		\item \( F \subseteq Q \) е \emph{множество от финални състояния}
		\item \( \Delta \subseteq Q \times \Sigma \times Q \) е \emph{релация на прехода}
	\end{itemize}
 
	Тройки от вида \( \langle q_1, m, q_2 \rangle \in \Delta \) наричаме \emph{преходи} и казваме, че започва състояние \( q_1 \), има етикет \( m \) и завършва в състояние \( q_2 \). Алтернативно, тези преходи обозначаваме като \( q_1 \to^m q_2 \).
\end{definition}

\begin{definition}  
	Нека \( \mathcal{A} \) е краен автомат. \emph{Разширена релация на прехода} \( \Delta^* \subseteq Q \times \Sigma^* \times Q \) дефинираме индуктивно:

	\begin{itemize}
		\item \( \langle q, \epsilon, q \rangle \in \Delta^* \) за всяко \( q \in Q \)
		\item \( \langle q_1, wa, q_2 \rangle \in \Delta^* \) за всяко \( q_1, q_2, q \in Q \), \( a \in \Sigma, w \in \Sigma^* \), ако \( \langle q_1, w, q \rangle \in \Delta^* \) и \( \langle q, a, q_2 \rangle \in \Delta \)
	\end{itemize}
\end{definition}

\begin{figure}[!htb]
	\centering
	% Start of code
% \begin{tikzpicture}[anchor=mid,>=latex',line join=bevel,]
\begin{tikzpicture}[>=latex',line join=bevel,]
  \pgfsetlinewidth{1bp}
%%
\pgfsetcolor{black}
  % Edge: 0 -> 1
  \draw [->] (41.38bp,33.346bp) .. controls (53.024bp,40.533bp) and (68.186bp,49.893bp)  .. (89.207bp,62.868bp);
  \definecolor{strokecol}{rgb}{0.0,0.0,0.0};
  \pgfsetstrokecolor{strokecol}
  \draw (65.5bp,56.0bp) node {b};
  % Edge: 0 -> 2
  \draw [->] (44.085bp,23.735bp) .. controls (64.334bp,25.44bp) and (95.778bp,28.188bp)  .. (123.0bp,31.0bp) .. controls (133.76bp,32.112bp) and (145.59bp,33.464bp)  .. (165.94bp,35.889bp);
  \draw (105.0bp,38.0bp) node {\(\epsilon\)};
  % Edge: 1 -> 1
  \draw [->] (98.266bp,89.037bp) .. controls (96.892bp,98.858bp) and (99.137bp,108.0bp)  .. (105.0bp,108.0bp) .. controls (108.66bp,108.0bp) and (110.92bp,104.43bp)  .. (111.73bp,89.037bp);
  \draw (105.0bp,115.0bp) node {a};
  % Edge: 1 -> 2
  \draw [->] (122.93bp,73.777bp) .. controls (130.82bp,73.967bp) and (140.18bp,73.252bp)  .. (148.0bp,70.0bp) .. controls (154.19bp,67.427bp) and (159.98bp,63.257bp)  .. (172.2bp,51.623bp);
  \draw (144.5bp,79.0bp) node {a};
  % Edge: 1 -> 2
  \draw [->] (120.65bp,62.769bp) .. controls (126.79bp,59.127bp) and (134.1bp,55.085bp)  .. (141.0bp,52.0bp) .. controls (145.96bp,49.782bp) and (151.38bp,47.722bp)  .. (166.36bp,42.716bp);
  \draw (144.5bp,59.0bp) node {b};
  % Edge: 2 -> 0
  \draw [->] (168.41bp,28.627bp) .. controls (156.67bp,21.673bp) and (139.47bp,12.77bp)  .. (123.0bp,9.0bp) .. controls (99.748bp,3.6755bp) and (72.92bp,7.7011bp)  .. (43.121bp,15.249bp);
  \draw (105.0bp,16.0bp) node {a};
  % Node: 0
\begin{scope}
  \definecolor{strokecol}{rgb}{0.0,0.0,0.0};
  \pgfsetstrokecolor{strokecol}
  \definecolor{fillcol}{rgb}{0.83,0.83,0.83};
  \pgfsetfillcolor{fillcol}
  \filldraw [opacity=1] (22.0bp,22.0bp) ellipse (18.0bp and 18.0bp);
  \draw (22.0bp,22.0bp) ellipse (22.0bp and 22.0bp);
  \draw (22.0bp,22.0bp) node {0};
\end{scope}
  % Node: 1
\begin{scope}
  \definecolor{strokecol}{rgb}{0.0,0.0,0.0};
  \pgfsetstrokecolor{strokecol}
  \draw (105.0bp,72.0bp) ellipse (18.0bp and 18.0bp);
  \draw (105.0bp,72.0bp) node {1};
\end{scope}
  % Node: 2
\begin{scope}
  \definecolor{strokecol}{rgb}{0.0,0.0,0.0};
  \pgfsetstrokecolor{strokecol}
  \draw (184.0bp,38.0bp) ellipse (18.0bp and 18.0bp);
  \draw (184.0bp,38.0bp) node {2};
\end{scope}
%
\end{tikzpicture}
% End of code

	\caption{Недетерминиран краен автомат}
\end{figure}

\begin{definition} 
	Нека \( \mathcal{A} = \langle \Sigma, Q, I, F, \Delta \rangle \) е краен автомат. \emph{Път} в \( \mathcal{A} \) наричаме крайна редица от преходи с дължина \( k > 0 \) 
	\[ \pi = q_0 \to^{a_1} q_1 \to^{a_2} \ldots \to^{a_k} q_k \] 
	където \( \langle q_{i-1}, a_i, q_i \rangle \in \Delta \) за \( i = 1 \ldots k \). Казваме, че \emph{пътят} започва от състояние \( q_0 \) и завършва в състояние \( q_k \). Елементите \( q_0,q_1, \ldots ,q_k \) наричаме \emph{състояния на пътя}, а думата \( w = a_1 a_2 \ldots a_k \) наричаме \emph{етикет на пътя}. \newline \emph{Успешен път} в автомата е \emph{път}, който започва от начално състояние и завършва във финално състояние.
\end{definition}

\begin{definition} 
	Нека \( \mathcal{A} \) е краен автомат. Множеството от етикети на всички успещни пътища в \( \mathcal{A} \) наричаме \emph{език на \( \mathcal{A} \)} и обозначаваме като \( L(\mathcal{A}) \). \[ L(\mathcal{A}) = \{ w \in \Sigma^* \mid \exists i \in I, f \in F : \langle i, w, f \rangle \in \Delta^* \} \]
\end{definition}

\begin{definition} 
	Нека \( \mathcal{A}_1 \) и \( \mathcal{A}_2 \) са крайни автомати. Казваме, че \( \mathcal{A}_1 \) е еквивалентен на \( \mathcal{A}_2 \) (\( \mathcal{A}_1 \equiv \mathcal{A}_2 \)), ако езиците им съвпадат (\( L(\mathcal{A}_1) = L(\mathcal{A}_2) \))
\end{definition}

\subsection{Детерминирани крайни автомати}

\begin{definition}
	Kраен автомат \( \mathcal{A} = \langle \Sigma, Q, I, F, \Delta \rangle \) е \emph{детерминиран}, ако:

	\begin{itemize}
		\item \( \mathcal{A} \) има единствено начално състояние \(I = \{q_0\}\).
		\item За всяко \( q_1 \in Q \) и символ \( a \in \Sigma \), съществува не повече от едно \( q_2 \in Q \), такова че \( \langle q_1, a, q_2 \rangle \in \Delta \).
	\end{itemize} 

	\noindent Иначе казано, релацията на прехода може да се представи като частична функция \( \delta: Q \times \Sigma \to Q \) и \emph{детерминираните автомати} можем преставим в следния вид \[ \mathcal{A}_D = \langle \Sigma, Q, q_0, F, \delta \rangle \]

	Предимството на \emph{детерминираните автомати} се изразява в това, че могат да разпознават дали дума \( w \) принадлежи на езика на автомата \( L(\mathcal{A}_D) \) за линейно време спрямо дължината ѝ - \( O(|w|) \).
\end{definition}

\begin{definition}
	Нека \( \mathcal{A}_D = \langle \Sigma, Q, q_0, F, \delta \rangle \) е \emph{детерминиран краен автомат}. \emph{Разширена функция на прехода} \( \delta^*: Q \times \Sigma^* \to Q \) дефинираме индуктивно:

	\begin{itemize}
		\item \( \delta^*(q, \epsilon) = q \)
		\item \( \delta^*(q, aw) = \delta^*(\delta(q, a), w) \), където \( a \in \Sigma, w \in \Sigma^* \)
	\end{itemize}
\end{definition}
asd
\begin{figure}[!htb]
	\centering
	% Start of code
% \begin{tikzpicture}[anchor=mid,>=latex',line join=bevel,]
\begin{tikzpicture}[>=latex',line join=bevel,]
  \pgfsetlinewidth{1bp}
%%
\pgfsetcolor{black}
  % Edge: 0 -> 0
  \draw [->] (14.317bp,42.991bp) .. controls (13.369bp,53.087bp) and (15.93bp,62.0bp)  .. (22.0bp,62.0bp) .. controls (25.889bp,62.0bp) and (28.337bp,58.342bp)  .. (29.683bp,42.991bp);
  \definecolor{strokecol}{rgb}{0.0,0.0,0.0};
  \pgfsetstrokecolor{strokecol}
  \draw (22.0bp,69.0bp) node {a};
  % Edge: 0 -> 1
  \draw [->] (43.022bp,29.416bp) .. controls (53.675bp,33.361bp) and (66.893bp,38.257bp)  .. (87.741bp,45.978bp);
  \draw (65.5bp,46.0bp) node {b};
  % Edge: 1 -> 2
  \draw [->] (119.63bp,62.906bp) .. controls (131.01bp,72.014bp) and (147.44bp,85.165bp)  .. (168.71bp,102.19bp);
  \draw (144.11bp,91.0bp) node {a};
  % Edge: 1 -> 3
  \draw [->] (123.04bp,52.0bp) .. controls (150.43bp,52.0bp) and (204.51bp,52.0bp)  .. (247.82bp,52.0bp);
  \draw (183.21bp,59.0bp) node {b};
  % Edge: 2 -> 3
  \draw [->] (197.95bp,102.65bp) .. controls (210.29bp,93.363bp) and (228.62bp,79.56bp)  .. (251.41bp,62.399bp);
  \draw (222.71bp,93.0bp) node {b};
  % Edge: 2 -> 4
  \draw [->] (197.25bp,124.43bp) .. controls (203.47bp,129.19bp) and (211.28bp,134.25bp)  .. (219.21bp,137.0bp) .. controls (223.88bp,138.62bp) and (228.98bp,139.63bp)  .. (244.14bp,140.96bp);
  \draw (222.71bp,145.0bp) node {a};
  % Edge: 3 -> 0
  \draw [->] (248.27bp,47.884bp) .. controls (221.74bp,41.645bp) and (168.72bp,29.994bp)  .. (123.0bp,25.0bp) .. controls (100.13bp,22.501bp) and (74.15bp,21.785bp)  .. (44.136bp,21.67bp);
  \draw (144.11bp,36.0bp) node {a};
  % Edge: 4 -> 2
  \draw [->] (247.16bp,128.57bp) .. controls (240.77bp,124.99bp) and (233.37bp,121.36bp)  .. (226.21bp,119.0bp) .. controls (221.48bp,117.44bp) and (216.31bp,116.29bp)  .. (201.28bp,114.07bp);
  \draw (222.71bp,126.0bp) node {b};
  % Edge: 4 -> 4
  \draw [->] (258.53bp,160.99bp) .. controls (257.58bp,171.09bp) and (260.14bp,180.0bp)  .. (266.21bp,180.0bp) .. controls (270.1bp,180.0bp) and (272.55bp,176.34bp)  .. (273.9bp,160.99bp);
  \draw (266.21bp,187.0bp) node {a};
  % Node: 0
\begin{scope}
  \definecolor{strokecol}{rgb}{0.0,0.0,0.0};
  \pgfsetstrokecolor{strokecol}
  \definecolor{fillcol}{rgb}{0.83,0.83,0.83};
  \pgfsetfillcolor{fillcol}
  \filldraw [opacity=1] (22.0bp,22.0bp) ellipse (18.0bp and 18.0bp);
  \draw (22.0bp,22.0bp) ellipse (22.0bp and 22.0bp);
  \draw (22.0bp,22.0bp) node {0};
\end{scope}
  % Node: 1
\begin{scope}
  \definecolor{strokecol}{rgb}{0.0,0.0,0.0};
  \pgfsetstrokecolor{strokecol}
  \draw (105.0bp,52.0bp) ellipse (18.0bp and 18.0bp);
  \draw (105.0bp,52.0bp) node {1};
\end{scope}
  % Node: 2
\begin{scope}
  \definecolor{strokecol}{rgb}{0.0,0.0,0.0};
  \pgfsetstrokecolor{strokecol}
  \draw (183.21bp,113.0bp) ellipse (18.0bp and 18.0bp);
  \draw (183.21bp,113.0bp) node {2};
\end{scope}
  % Node: 3
\begin{scope}
  \definecolor{strokecol}{rgb}{0.0,0.0,0.0};
  \pgfsetstrokecolor{strokecol}
  \draw (266.21bp,52.0bp) ellipse (18.0bp and 18.0bp);
  \draw (266.21bp,52.0bp) node {3};
\end{scope}
  % Node: 4
\begin{scope}
  \definecolor{strokecol}{rgb}{0.0,0.0,0.0};
  \pgfsetstrokecolor{strokecol}
  \draw (266.21bp,140.0bp) ellipse (18.0bp and 18.0bp);
  \draw (266.21bp,140.0bp) ellipse (22.0bp and 22.0bp);
  \draw (266.21bp,140.0bp) node {4};
\end{scope}
%
\end{tikzpicture}
% End of code

	\caption{Детерминиран краен автомат}
\end{figure}

\begin{theorem}
	За всеки краен автомат \(\mathcal{A} = \langle \Sigma, Q, I, F, \Delta \rangle \), съществува еквивалентен на него, детерминиран краен автомат \( \mathcal{A}_D \).
	\begin{proof}
		Строим \emph{детерминиран краен автомат} \( \mathcal{A}_D = \langle \Sigma, 2^Q, I, F_D, \delta \rangle \), където
		\begin{itemize}
			\item \( F_D = \{ S \in 2^Q \mid S \cap F \neq \emptyset \} \)
			\item \( \delta(S,a) = \{ q \in Q \mid \exists q_1 \in S : \langle q_1, a, q \rangle \in \Delta \} \)
		\end{itemize}
	\end{proof}
\end{theorem}

\subsection{Крайни преобразуватели}

\begin{definition}
	\emph{Краен преобразувател} дефинираме като петорка \( \mathcal{T} = \langle \Sigma \times \Gamma, Q, I, F, \Delta \rangle \), където

	\begin{itemize}
		\item \( \Sigma, \Gamma \) са \emph{крайни азбуки от символи}
		\item \( Q \) е \emph{крайно множество от състояния}
		\item \( I \subseteq Q \) е \emph{множество от начални състояния}
		\item \( F \subseteq Q \) е \emph{множество от финални състояния}
		\item \( \Delta \subseteq Q \times (\Sigma^* \times \Gamma^*) \times Q \) е \emph{релация на прехода}
	\end{itemize}
 
	Тройки от вида \( \langle q_1, \langle w, m \rangle, q_2 \rangle \in \Delta \) наричаме \emph{преходи} и казваме, че започва състояние \( q_1 \), има етикет по горната лента \( w \) и по долната лента \( m \) и завършва в състояние \( q_2 \). Алтернативно, тези преходи обозначаваме като \( q_1 \to^w_m q_2 \).
\end{definition}

\begin{figure}[!htb]
	\centering
	% Start of code
% \begin{tikzpicture}[anchor=mid,>=latex',line join=bevel,]
\begin{tikzpicture}[>=latex',line join=bevel,]
  \pgfsetlinewidth{1bp}
%%
\pgfsetcolor{black}
  % Edge: 0 -> 1
  \draw [->] (36.835bp,66.461bp) .. controls (43.628bp,73.272bp) and (52.411bp,80.407bp)  .. (62.0bp,84.0bp) .. controls (70.197bp,87.071bp) and (79.618bp,87.58bp)  .. (98.159bp,85.873bp);
  \definecolor{strokecol}{rgb}{0.0,0.0,0.0};
  \pgfsetstrokecolor{strokecol}
  \draw (70.945bp,94.0bp) node {a/d};
  % Edge: 0 -> 2
  \draw [->] (42.234bp,41.004bp) .. controls (48.469bp,38.253bp) and (55.463bp,35.34bp)  .. (62.0bp,33.0bp) .. controls (70.433bp,29.982bp) and (79.79bp,27.161bp)  .. (98.033bp,22.16bp);
  \draw (70.945bp,40.0bp) node {b/d};
  % Edge: 1 -> 0
  \draw [->] (99.425bp,74.306bp) .. controls (93.413bp,71.481bp) and (86.413bp,68.392bp)  .. (79.89bp,66.0bp) .. controls (71.374bp,62.877bp) and (61.955bp,60.003bp)  .. (43.517bp,54.95bp);
  \draw (70.945bp,73.0bp) node {b/\(\epsilon\)};
  % Edge: 2 -> 0
  \draw [->] (99.067bp,10.797bp) .. controls (88.259bp,6.9443bp) and (73.813bp,3.9331bp)  .. (62.0bp,9.0bp) .. controls (54.255bp,12.322bp) and (47.322bp,18.021bp)  .. (34.816bp,31.943bp);
  \draw (70.945bp,16.0bp) node {c/\(\epsilon\)};
  % Node: 0
\begin{scope}
  \definecolor{strokecol}{rgb}{0.0,0.0,0.0};
  \pgfsetstrokecolor{strokecol}
  \definecolor{fillcol}{rgb}{0.83,0.83,0.83};
  \pgfsetfillcolor{fillcol}
  \filldraw [opacity=1] (22.0bp,50.0bp) ellipse (18.0bp and 18.0bp);
  \draw (22.0bp,50.0bp) ellipse (22.0bp and 22.0bp);
  \draw (22.0bp,50.0bp) node {0};
\end{scope}
  % Node: 1
\begin{scope}
  \definecolor{strokecol}{rgb}{0.0,0.0,0.0};
  \pgfsetstrokecolor{strokecol}
  \draw (115.89bp,82.0bp) ellipse (18.0bp and 18.0bp);
  \draw (115.89bp,82.0bp) node {1};
\end{scope}
  % Node: 2
\begin{scope}
  \definecolor{strokecol}{rgb}{0.0,0.0,0.0};
  \pgfsetstrokecolor{strokecol}
  \draw (115.89bp,18.0bp) ellipse (18.0bp and 18.0bp);
  \draw (115.89bp,18.0bp) node {2};
\end{scope}
%
\end{tikzpicture}
% End of code

	\caption{Краен преобразувател}
\end{figure}

\begin{definition}
	TODO: \emph{Път и успешен път в краен преобразувател}
\end{definition}

\begin{definition}
	TODO: \emph{Подлежащ автомат на краен преобразувател}
\end{definition}

\subsection{Регулярни езици и релации}

\begin{definition} 
	\emph{Регулярен език} е множество над крайна азбука \( \Sigma \), което дефинираме индуктивно:
	\begin{itemize}
		\item \( \emptyset \) е регулярен език.
		\item ако \( a \in \Sigma \), то \( \{ a \} \) е регулярен език.
		\item ако \( L_1, L_2 \in \Sigma^* \) са регулярни езици, то 
		\begin{itemize}
			\item \( L_1 \cup L_2 \) (обедниение)
			\item \( L_1 \cdot L_2 = \{ a \cdot b \mid a \in L_1, b \in L_2 \} \) (конкатенация)
			\item \( L_1^* = \bigcup_{i=0}^{\infty}L^i \) (звезда на Клини)
		\end{itemize}
		 са също регулярни езици.
		\item Не съществуват други регулярни езици
	\end{itemize}
\end{definition}

\begin{definition} 
	\emph{Регулярна n-орна стрингова релация} дефинираме индуктивно като множество от n-орки над крайни азбуки \( \Sigma_1, \dots, \Sigma_n \):
	\begin{itemize}
		\item \( \emptyset \) е регулярна релация.
		\item Ако \( a \in \Sigma_1 \times ... \times \Sigma_n \), то \( \{ a \} \) е регулярна релация.
		\item Ако \( R_1, R_2 \) са регулярни релации, то:
		\begin{itemize}
			\item \( R_1 \cup R_2 \) (обединение)
			\item \( R_1 \cdot R_2 = \{ a \cdot b \mid a \in R_1, b \in R_2 \} \) (конкатенация)
			\item \( R_1^* = \bigcup_{i=0}^{\infty}R^i \) (звезда на Клини)
		\end{itemize}
		са също регулярни релации.
		\item Не съществуват други регулярни релации.
	\end{itemize}
\end{definition}

\begin{definition} 
	\emph{Регулярен израз} наричаме дума над крайна азбука \( \Sigma \cup \{ (, ), \vert, * \} \)
	\begin{itemize}
		\item \( \epsilon \) е регулярен израз.
		\item Ако \( a \in \Sigma \), то \( a \) е регулярен израз.
		\item Ако \( E_1, E_2 \) са регулярни изрази, то \( E_1 \mid E_2 \) и \( E_1 E_2 \), \( E_1^* \) също са регулярни изрази.
		\item Не съществуват други регулярни изрази.
	\end{itemize}
	Всеки регулярен израз има сътветстващ регулярен език:
	\begin{itemize}
		\item \( L(\epsilon) = \emptyset \)
		\item \( L(a) = \{ a \}, a \in \Sigma \)
		\item \( L(E_1 \mid E_2) = L(E_1) \cup L(E_2) \)
		\item \( L(E_1 E_2) = L(E_1) \cdot L(E_2) \)
		\item \( L(E^*) = L(E)^* \)
	\end{itemize}
\end{definition}

\begin{example}
	TODO: Пример за регулярен израз
\end{example}

\begin{theorem} {\emph{(Клини)}}
	За всеки регулярен израз \( E \) съществува краен автомат \( \mathcal{A} \), за който \( L(E) = L(\mathcal{A})\)
\end{theorem}

\begin{theorem}
	Всяка двоична регулярни стрингова релациа можемогат да се изрази, чрез класически краен преобразувател.
\end{theorem}

\begin{definition}
	Регулярна стрнгова функция наричаме регулярна стрингова релация, която е частична функция. 
\end{definition}

\begin{example}
	TODO: Пример за регулярна стрингова релация
\end{example}

\subsection{Свойства на регулярните езици и релации}

\subsection{Бимашини}

\begin{definition}
	 \emph{Бимашина} дефинираме като тройка \( \mathcal{B} = \langle \mathcal{A}_L, \mathcal{A}_R, \psi \rangle \), където

	\begin{itemize}
		\item \( \mathcal{A}_L = \langle \Sigma, Q_L, s_L, Q_L, \delta_L \rangle \) и \( \mathcal{A}_R = \langle \Sigma, Q_R, s_R, Q_R, \delta_R \rangle \) са \emph{детерминирани крайни автомати} и ги наричаме съответно \emph{ляв и десен автомат} на бимашината. Всички състояния на тези автомати са финални.
		\item \( \psi:(Q_L \times \Sigma \times Q_R) \to \Sigma^* \) e частична функция, която наричаме \emph{изходна функция}.
	\end{itemize}
\end{definition}

\begin{definition} 
	Нека \( \mathcal{B} = \langle \mathcal{A}_L, \mathcal{A}_R, \psi \rangle \) е класическа бимашина, \( \Sigma \) е азбуката на на автоматите \( \mathcal{A}_L \) и \( \mathcal{A}_R \) и \( w = a_1a_2 \dots a_k \in \Sigma^* \) \( (k \geq 0) \), дума и \( a_i \in \Sigma \) \( (1 \leq i \leq k) \) са букви. Ако \( \delta_L^*(a_1a_2 \ldots a_k) \) и \( \delta_R^*(a_ka_{k-1} \dots a_1) \) са дефинирани, то можем да получим двата пътя:
	\[ \pi_L = l_0 \to^{a_1} l_1 \to^{a_2} \ldots l_{k-1} \to^{a_k} l_k \]
	\[ \pi_R = r_0 \leftarrow^{a_1} r_1 \leftarrow^{a_2} \ldots r_{k-1} \leftarrow^{a_k} r_k \]
	Където \( \pi_L \) и \( \pi_R \) са пътища в съответно левия и десния автомат и думата \( w \) се разпознава от \( \mathcal{A}_L \) в посока от ляво на дясно, а от \( \mathcal{A}_R \), съответно от дясно на ляво. Ако за всички тройки \( \langle l_{i-1}, a_i, r_i \rangle \), изходната функция \( \psi(l_{i-1}, a_i, r_i) \) е дефинирана, то двойката пътища \( \langle \pi_L, \pi_R \rangle \) наричаме \emph{успешно изпълнение} на \( \mathcal{B} \) с етикет \( w = a_1a_2 \dots a_k \) и \emph{изход}
	\[ \mathcal{O_B}(w) = \psi(l_0, a_1, r_1) \cdot \psi(l_1, a_2, r_2) \cdot \dots \cdot \psi(l_{k-1}, a_k, r_k) \]
	\( \mathcal{O_B} \) наричаме \emph{изходна функция на бимашината} и казваме, че бимашината \emph{превежда} \(w\) в \(m\), ако \( \mathcal{O_B}(w) = m \), където \(m\) е резултат от конкатенация на всички \( \psi(l_{i-1}, a_i, r_i) \) \( 1 \leq i \leq k \).
\end{definition}

\noindent Бимашината чете входната дума и за всеки символ извежда дума над азбуката си. На всяка стъпка изведената от изходната функция \( \psi \) дума зависи от входния символ и двете състояния в които биха преминали левият и десният автомат, четейки входа съответно от ляво на дястно и от дясно на ляво. Крайният резултат е конкатенацията на всички така изведени думи.

\begin{example}{\emph{(Бимашина и изпълнение)}}
	TODO: Пример бимашина
\end{example}

\begin{definition}
	Нека \( \mathcal{B} = \langle \mathcal{A}_L, \mathcal{A}_R, \psi \rangle \) е бимашина. \emph{Разширената изходна функция \( \psi^* \)} дефинираме индуктивно:

	\begin{itemize}
		\item \( \psi^*(l, \epsilon, r) = \epsilon \) за всяко \( l \in Q_L, r \in Q_R \)
		\item \( \psi^*(l, wa, r) = \psi^*(l, w, \delta_R(r, a)) \cdot \psi(\delta^*_L(l, w), a, r) \), за \( l \in Q_L, r \in Q_R, w \in \Sigma^*, a \in \Sigma \)
	\end{itemize}
\end{definition}

\begin{definition}
	Бимашина \( \mathcal{B} = \langle \mathcal{A}_L, \mathcal{A}_R, \psi \rangle \) наричаме \emph{тотална}, ако функциите на прехода \( \delta_L: Q_L \times \Sigma \to Q_L \) и \( \delta_R: Q_R \times \Sigma \to Q_R \) на левия и десния автомат съответно, както и функцията на изхода \( \psi: (Q_L \times \Sigma \times Q_R) \to \Sigma^* \) са \emph{тотални}.
\end{definition}

\section{Правила за заместване чрез регулярни релации}

\begin{definition}
	Нека \( \Sigma_1, \Sigma_2 \) са крайни азбуки. \emph{Двоична стрингова релация} \( R \subseteq \Sigma_1^* \times \Sigma_2^* \) наричаме множество от двойки \( \langle u, v \rangle \), където \( u \in \Sigma_1^* \) и \( v \in \Sigma_2^* \).
\end{definition}

За двоичните стрингови релации можем да си мислим като множество от преводи, като на пример за двойката \( \langle u, v \rangle \in R \) казваме, че думата \( u \) се превежда като \( v \).

\begin{definition}
	Нека \( R_1, R_2, \in \Sigma^* \times \Sigma^* \) са \emph{двоични стрингови релации}. \emph{Конкатенацията} на \( R_1 \) и \( R_2 \) дефинираме както следва
	\[ R_1 \cdot R_2 = \{ \langle u_1 \cdot v_1, u_2 \cdot v_2 \rangle \mid \langle u_1, u_2 \rangle \in R_1, \langle v_1, v_2 \rangle \in R_2 \} \]
\end{definition}

\begin{definition}
	\emph{Двоична стрингова релация} \( R \subseteq L_1 \times L_2 \) e \emph{регулярна}, ако \(L_1\) и \(L_2\) са регулярни езици.
\end{definition}

\begin{definition}
	\emph{Правило за заместване} представяме във вида
	\[ E \to \beta \]
	където \( E \) е регулярен израз над крайна азбука \( \Sigma \), а \( \beta \in \Sigma^* \) е дума.
	\emph{Приложение на правилото върху текст} \( t \in \Sigma^* \) представлява заместването на поднизовете на \( t \), които са в езика \( L(E) \) с \( \beta \).
\end{definition}

\begin{figure}[!htb]
	\centering
	\begin{tikzpicture}
		\node[font=\Large] {
			\( a_1a_2 \dots a_i \dots a_{i+k} \dots a_{j} \dots a_{j+l} \dots a_n \)
		};
		\draw [decorate,decoration={brace,amplitude=5pt,raise=0pt},yshift=0pt]
		(-2.75,0.4) -- (-0.25,0.4) node [black,midway,xshift=0cm,yshift=0.5cm] {\footnotesize $ \in L(E) $};
		\draw [decorate,decoration={brace,amplitude=5pt,mirror,raise=0pt},yshift=0pt]
		(-2.75,-0.3) -- (-0.25,-0.3) node [black,midway,xshift=0cm,yshift=-0.5cm] {\footnotesize $ \beta $};
		\draw [decorate,decoration={brace,amplitude=5pt,raise=0pt},yshift=0pt]
		(0.8,0.4) -- (3.15,0.4) node [black,midway,xshift=0cm,yshift=0.5cm] {\footnotesize $ \in L(E) $};
		\draw [decorate,decoration={brace,amplitude=5pt,mirror,raise=0pt},yshift=0pt]
		(0.8,-0.3) -- (3.15,-0.3) node [black,midway,xshift=0cm,yshift=-0.5cm] {\footnotesize $ \beta $};
	\end{tikzpicture}
	\caption{Приложение на правило на заместване}
\end{figure}

\emph{Правила за заместване} можем да представим като \emph{регулярни стрингови релации}, използвайки единствено алгебрата на регулярните множества и релации. Те могат да се реализират програмно чрез крайни преобразуватели. \cite{Kaplan&Kay94}

\begin{definition}
	Нека разгледаме текст \( t \in \Sigma^* \). \emph{Kонтекст на заместване} наричаме тройка \( \langle u,v,w \rangle \), където \( t = uvw \) и \( u, w \in \Sigma^* \). Също така \( u \) и \( w \) наричаме съответно \emph{префикс} и \emph{суфикс} на контекста, докато \( v \) наричаме \emph{фокус}.
\end{definition}

\begin{example}
	Нека разгледаме правилото \( ab|bc \to d \). След приложението му над текста \emph{abb}, ще получим \emph{db}, като \( \langle \epsilon, ab, b \rangle \) е единствения контекст на заместване. Приложението на правилото над \emph{abbacbca} ще доведе до \emph{dbacda} с два контекста на заместване \( \langle \epsilon, ab, bacbca \rangle \) и \( \langle abbac, bc, a \rangle \).
\end{example}

\subsection{Разрешаване на многозначности}

При прилагане на правило за заместване могат да възникнат многозначности в случаите, в които два контекста имат застъпващи се фокуси.

\begin{definition}
	Два контекста на заместване \( \langle u_1,v_1,w_1 \rangle \) и \( \langle u_2,v_2,w_2 \rangle \) за даден текст \( t \) \emph{се застъпват}, ако \( u_1 < u_2 < u_1v_1 \). Израза \( u_1 < u_2 \) четем като \( u_1 \) е \emph{префикс} на \( u_2 \).
\end{definition}

\begin{figure}[!htb]
	\centering
	\begin{tikzpicture}
	\draw (0,0) rectangle (3,1); % u_1
	\draw (0,0) rectangle (5,1); % u_2
	\draw[pattern=horizontal lines] (3,0) rectangle (7,1); % v_1
	\draw[pattern=vertical lines] (5,0) rectangle (8,1); % v_2
	\draw (7,0) rectangle (11,1); % w_1
	\draw (8,0) rectangle (11,1); % w_2
	\draw [decorate,decoration={brace,amplitude=10pt,raise=3pt},yshift=0pt]
		(0,1) -- (3,1) node [black,midway,xshift=0cm,yshift=0.75cm] {\footnotesize $u_1$};
	\draw [decorate,decoration={brace,amplitude=10pt,mirror,raise=3pt},yshift=0pt] 
		(0,0) -- (5,0) node [black,midway,xshift=0cm,yshift=-0.75cm] {\footnotesize $u_2$};
	\draw [decorate,decoration={brace,amplitude=10pt,raise=3pt},yshift=0pt]
		(3,1) -- (7,1) node [black,midway,xshift=0cm,yshift=0.75cm] {\footnotesize $v_1$};
	\draw [decorate,decoration={brace,amplitude=10pt,mirror,raise=3pt},yshift=0pt]
		(5,0) -- (8,0) node [black,midway,xshift=0cm,yshift=-0.75cm] {\footnotesize $v_2$};
	\draw [decorate,decoration={brace,amplitude=10pt,raise=3pt},yshift=0pt]
		(7,1) -- (11,1) node [black,midway,xshift=0cm,yshift=0.75cm] {\footnotesize $w_1$};
	\draw [decorate,decoration={brace,amplitude=10pt,mirror,raise=3pt},yshift=0pt]
		(8,0) -- (11,0) node [black,midway,xshift=0cm,yshift=-0.75cm] {\footnotesize $w_2$};
	\end{tikzpicture}
	\caption{Застъпващи се контексти}
\end{figure}

\begin{example}
	Нека разгледаме правилото \( ab|bc \to d \) приложено над текста \( t = aabcb \). Получаваме контекстите \( \langle a, ab, cb \rangle \) и \( \langle aa, bc, b \rangle \), които очвевидно се застъпват и съответно стигаме до две различни валидни замествания \emph{adcb} и \emph{aadb}.
\end{example}

\begin{figure}[!htb]
	\centering
	\begin{tikzpicture}
	\draw (0,0) rectangle (3,1); % u_1
	\draw (0,0) rectangle (3,1); % u_2
	\draw[pattern=horizontal lines] (3,0) rectangle (8,1); % v_1
	\draw[pattern=vertical lines] (3,0) rectangle (6,1); % v_2
	\draw (8,0) rectangle (11,1); % w_1
	\draw (6,0) rectangle (11,1); % w_2
	\draw [decorate,decoration={brace,amplitude=10pt,raise=3pt},yshift=0pt]
		(0,1) -- (3,1) node [black,midway,xshift=0cm,yshift=0.75cm] {\footnotesize $u_1$};
	\draw [decorate,decoration={brace,amplitude=10pt,mirror,raise=3pt},yshift=0pt] 
		(0,0) -- (3,0) node [black,midway,xshift=0cm,yshift=-0.75cm] {\footnotesize $u_2$};
	\draw [decorate,decoration={brace,amplitude=10pt,raise=3pt},yshift=0pt]
		(3,1) -- (8,1) node [black,midway,xshift=0cm,yshift=0.75cm] {\footnotesize $v_1$};
	\draw [decorate,decoration={brace,amplitude=10pt,mirror,raise=3pt},yshift=0pt]
		(3,0) -- (6,0) node [black,midway,xshift=0cm,yshift=-0.75cm] {\footnotesize $v_2$};
	\draw [decorate,decoration={brace,amplitude=10pt,raise=3pt},yshift=0pt]
		(8,1) -- (11,1) node [black,midway,xshift=0cm,yshift=0.75cm] {\footnotesize $w_1$};
	\draw [decorate,decoration={brace,amplitude=10pt,mirror,raise=3pt},yshift=0pt]
		(6,0) -- (11,0) node [black,midway,xshift=0cm,yshift=-0.75cm] {\footnotesize $w_2$};
	\end{tikzpicture}
	\caption{Застъпващи се контексти с еднакво начало}
\end{figure}

\begin{example}
	Друг вид многозначност може да получим, когато фокусите на два контекста имат еднакво начало. Например, ако приложим правилото \(a^+ \to d \) над тескт \( t = aa \), може получим превода \emph{dd}, на който отговарят контекстите \( \langle a, a, \epsilon \rangle \) и \( \langle \epsilon, a, a \rangle \), и \emph{d} с контекст \( \langle \epsilon, aa, \epsilon \rangle \).
\end{example}

\begin{definition}
	\emph{Дефинираме следните функции над множества от контексти. OVER, MOST, LONGEST}
\end{definition}

\begin{definition}
	\emph{Множество на най-левите и най-дълги контексти} LML(A)
\end{definition}

\begin{proposition}
	LML(A) съдържа само незастъпващи се контексти.
	\begin{proof}
		...
	\end{proof}
\end{proposition}

\begin{definition}
	\emph{Приложение на правило за заместване над дума в текст под стратегията LML}
\end{definition}

\begin{corollary}
	Ако релацията е функционална, то LML е функция.
\end{corollary}

\subsection{Токенизация чрез регулярни релации}

\emph{Токенизираща релация} за даден език дефинираме като комбинация на \emph{правила за заместване}. Тези правила биват два вида - такива, които \emph{маркират} дадена лексема, или \emph{нормализиращи правила}, които служат за предварителна обработка входния текст, като на пример за премахване на излишни символи.

\begin{example}{\emph{Лексическа граматика на аритметичен израз от естествени числа.}}
	\begin{enumerate}
		\item \verb/WHITESPACE+/ \( \to \epsilon \)
		\item \verb/SoT/ \verb/[0-9]+/ \( \to \dots \) \verb/EoT/
		\item \verb/SoT/ \verb/+|-|*|// \( \to \dots \) \verb/EoT/
	\end{enumerate}
\end{example}

Правила 2 и 3 маркират лексемите съответно за естествено число и аритметичен оператор като поставят символ за граница на лексема \( \verb/TB/ \). Правило 1 е нормализиращо правило, което премахва интервалите от входната дума, тъй като те не носят значение за аритметичния израз.

Всяко от тези правила само по себе си представлява регулярна релация, като релацията за лексически анализ в конкретния случай получаваме като резултат от композицията на нормализиращото правило с обединението на лексическите правила. С други думи, ако всяко правило \(i\) представим като регулярна релация \(R_i\), токенизиращата релация под стратегията най-ляво-най-дълго срещане можем да представим както следва

\[ R = R^{LML}(R_1) \circ R^{LML}(R_2 \cup R_3) \]

Стратегията най-ляво-най-дълго срещане приложена над функционални релации, каквито са правилата в лексическата граматика, гарантира, че получената релация също е функционална. Това означава, че процесът на токенизация винаги ще има най-много един резултат.

\[ R(\verb/"2 + 14"/) = \verb/"2 TB + TB 14 TB"/ \]
\[ R(\verb/"42-15*5"/) = \verb/"42 TB - TB 15 TB * TB 5 TB"/ \]

\begin{theorem}
	TODO: Еквивалентност на регулярните стрингови функции и бимашините ...
\end{theorem}

\section{Реализация}

\subsection{Дизайн}
\subsection{Парсер на регулярен израз}
\subsection{Конструкция на бимашина}
\subsection{Процедура за токенизация}

\pagebreak

\begin{thebibliography}{9}
	\bibitem{Kaplan&Kay94} 
	Kaplan, Ronald and Kay, Martin. 1994.
	\textit{Regular models of phonological rule systems.}
	Computational Linguistics 20(3):331-378

	\bibitem{Karttunen96}
	Karttunen, Lauri. 1996.
	\textit{Directed Replacement.}
	
\end{thebibliography}	

\end{document}
